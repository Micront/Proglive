%\documentclass[12pt]{article}
\documentclass[a4paper,12pt]{article}
\usepackage[utf8]{inputenc}
\usepackage[T2A]{fontenc}
\usepackage[english,russian]{babel}
\usepackage[left=3cm,right=1.5cm,top=2cm,bottom=2cm]{geometry}
\usepackage{indentfirst}
\usepackage{pscyr}
\usepackage{amssymb,latexsym,amsmath,amscd}

%межстрочный интервал
\renewcommand{\baselinestretch}{1.25}

\title{Java 2. Профессиональное программирование}
\author{Архангельский Дмитрий}
\date{}

\begin{document}
\maketitle

Курс предназначен для тех, кто хочет узнать о продвинутых возможностях языка Java и получить практический опыт в написании приложений. Нужно разбираться в базовых понятиях программирования и иметь начальный опыт работы с Java. Курс охватывает все основные темы, необходимые для профессиональной работы.

Вы научитесь создавать сложные, многокомпонентные приложения с графическим интерфейсом. Вы будете знать, как работать с базами данных и как передавать данные по сети, узнаете об отличительной черте Java - reflection. В курсе будут рассмотрены темы, связанные с объектно-ориентированным программированием: внутренние и анонимные классы, шаблоны проектирования, обобщенное программирование. Часть курса будет посвящена созданию графического интерфейса и графической подсистеме Java. Вы научитесь писать многопоточный код, узнаете о методах синхронизации и познакомитесь с библиотекой java.util.concurrent. 

Будет подробно разобран процесс разработки программного обеспечения: постановка задачи, архитектура приложения, тестирование. Вы научитесь подключать сторонние библиотеки, использовать логирование и отладчик. Вы самостоятельно разработаете несколько приложений по основным темам курса: менеджер расходов, многопользовательский чат, редактор фотографий.

\subsection*{План курса}
\begin{enumerate}
\item Generics - Параметризация и обобщенное программирование
    \begin{itemize}
        \item назначение Java Generics
        \item generic-контейнеры java.util
        \item ограничения и маски (super, extends)
        \item использование wildcards
        \item generic методы и классы
        \item особенности наследования
    \end{itemize}
    
\item Базы данных
    \begin{itemize}
        \item реляционные базы данных
        \item язык запросов SQL
        \item операторы select, insert, update, delete
        \item jdbc - database connection
        \item запросы в базу данных, обработка результата
        \item создание и удаление таблиц
    \end{itemize}
    
\item Swing. Графический интерфейс
    \begin{itemize}
        \item JFrame - оконная модель
        \item элементы интерфейса: кнопки, поля ввода, меню, списки, надписи
        \item компоновка элементов, менеджер компоновки
        \item примеры основных Layout Manager: FlowLayout, BorderLayout, BoxLayout, GridBagLayout
        \item обработка событий
        \item AWT - рисование, основные понятия
    \end{itemize}

\item Внутренние классы
    \begin{itemize}
        \item понятие внутреннего класса
        \item применение на практике
        \item анонимные и локальные классы
        \item вложенные классы
        \item лямбда
        \item интерфейсы и типы
    \end{itemize}

\item Средства ввода-вывода
    \begin{itemize}
        \item обзор java.io
        \item байтовые и символьные потоки
        \item буферизованные потоки
        \item сетевое взаимодействие, сокеты
        \item сетевой чат
        \item сериализация объектов
    \end{itemize}

\item Многопоточность
    \begin{itemize}
        \item понятие потока, разделяемая память
        \item потоки в Java. Класс Thread, интерфейс Runnable
        \item управление потоком, состояния потока
        \item синхронизация: lock, synchronized, wait/notify
        \item deadlock - взаимная блокировка
        \item concurrency collections
        \item многопользовательский сетевой чат
    \end{itemize}
    
\item Обзор средств разработки. Внешние библиотеки. Тестирование
    \begin{itemize}
        \item уровни логирования
        \item подключение внешних библиотек
        \item виды тестирования
        \item юнит-тесты, библиотека JUnit
        \item разработка через тестирование (TDD)
    \end{itemize}

\item Reflection, annotations
    \begin{itemize}
        \item понятие reflection
        \item класс Class
        \item метаданные классов, доступ к полям и методам
        \item аннотации - примеры использования
        \item написание собственных аннотаций
    \end{itemize}
    
\end{enumerate}
\end{document}