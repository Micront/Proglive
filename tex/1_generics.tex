\documentclass[a4paper,12pt]{extreport}
\usepackage[utf8]{inputenc}
\usepackage[T2A]{fontenc}
\usepackage[english,russian]{babel}
\usepackage[left=3cm,right=1.5cm,top=2cm,bottom=2cm]{geometry}
\usepackage{indentfirst}
\usepackage{pscyr}
\usepackage{amssymb,latexsym,amsmath,amscd}

% listings
\usepackage{listings}
\usepackage{color}

\definecolor{dkgreen}{rgb}{0,0.6,0}
\definecolor{gray}{rgb}{0.5,0.5,0.5}
\definecolor{mauve}{rgb}{0.58,0,0.82}

\lstset{frame=tb,
  language=Java,
  aboveskip=3mm,
  belowskip=3mm,
  showstringspaces=false,
  columns=flexible,
  basicstyle={\small\ttfamily},
  numbers=none,
  numberstyle=\tiny\color{gray},
  keywordstyle=\color{blue},
  commentstyle=\color{dkgreen},
  stringstyle=\color{mauve},
  breaklines=true,
  breakatwhitespace=true
  tabsize=2
}

%межстрочный интервал
\renewcommand{\baselinestretch}{1.25}
%значки больше-меньше
\renewcommand{\ge}{\geqslant}
\renewcommand{\le}{\leqslant}

\renewcommand*\thesection{\arabic{section}}

\begin{document}
\section{Параметризация и обобщенное программирование}
Параметризация позволяет использовать тип как параметр при определении класса, интерфейса или метода. Это позволяет переиспользовать один код для различных типов входных параметров.

Также параметризированный код дает дополнительные преимущества:
\begin {itemize}
    \item Проверка типов на стадии компиляции. Лучше исправлять compile-time error, чем runtime error.
    \item Нет необходимости вручную приводить типы.
        \begin{lstlisting}
        // without generics
        List list = new ArrayList();
        list.add("Hello");
        String s = (String) list.get(0);
        
        // generics
        List<String> list = new ArrayList<>();
        list.add("Use generics everywhere!");
        String s = list.get(0);
        \end{lstlisting}
    \item Обобщенные алгоритмы. Java Collections невозможно представить без параметризации - контейнеры могут хранить любой тип данных.
\end{itemize}

\subsection*{Объявление}
Компилятору можно принудительно указать, с каким типом данных работает коллекция. В качестве параметра нельзя использовать примитивные типы данных. 
\begin{lstlisting}
List<String> strings = new ArrayList<String>();
List<String> strings = new ArrayList<>(); 
StringBuilder builder = new StringBuilder();
for (String s : strings) {
    builder.append(s);
}

Map<Manager, List<Employee>> staff = new HashMap<>();
// Look through the map
for (Map.Entry<Manager, List<Employee>> entry : staff.entrySet()) {
    Manager manager = entry.getKey();
    List<Employee> employees = entry.getValue();
}
\end{lstlisting}

Начиная с версии Java 7 можно использовать diamond operator <> в правой части выражения, и компилятор самостоятельно выведет правильный тип.

Чтобы определить параметризованный класс используется синтаксис Some<T1, T2, T3,..> , где Ti - это тип. По стандартному соглашению в качестве имен параметров используются буквы, передающие смысл параметра:
\begin{itemize}
    \item E - Element
    \item K - Key
    \item N - Number
    \item T - Type
    \item V - Value
    \item S,U,V etc. - 2nd, 3rd, 4th types
\end{itemize}

В коде класса вместо конкретного типа используется параметр.
\begin{lstlisting}
/**
 * Generic version of the Box class.
 * @param <T> the type of the value being boxed
 */
public class Box<T> {
    private T t;

    public void set(T t) { this.t = t; }
    public T get() { return t; }
}

/**
 * @param <K> the type of key 
 * @param <V> the type of value
 */
public interface Pair<K, V> {
    K getKey();
    V getValue();
}
\end{lstlisting}

Чтобы объявить параметризованный метод, в его сигнатуре следует указать используемые типы перед типом возвращаемого значения. 

\begin{lstlisting}
public class Util {
    // Generic static method
    public static <K, V> boolean compare(Pair<K, V> p1, Pair<K, V> p2) {
        return p1.getKey().equals(p2.getKey()) &&
               p1.getValue().equals(p2.getValue());
    }
}
\end{lstlisting}

\subsection*{Ограничения и маски}
При использовании параметризации компилятор проверяет точное совпадение типов. Однако иногда нужно расширить функционал и позволить обрабатывать объекты дочерних или родительских классов. 


\textbf{T extends Some} - элементы класса Some и всех его потомков. 

\textbf{T super Some} - элементы класса Some и всех его родителей.

\begin{lstlisting}
// Number and all sub-types are allowed
public static <T extends Number> T getFirst(List<T> numbers) {
    T t = numbers.get(0);
    return t;
}

// Integer and all supertypes are allowed
pubic static <T super Integer> void fillContainer(List<T> list) {
    Integer i = 10;
    list.add(i);
}

public static void test() {
    // class Integer is a subtype of Number
    List<Integer> integers = Arrays.asList{1, 2, 3, 4};
    Integer first = getFirst(integers);
    
    // Number is a supertype of Integer
    List<Number> numbers = new ArrayList<>();
    fillContainer(numbers);
}
\end{lstlisting}

\subsection*{Wildcards}
Если нам не важно, какой тип используется и мы не возвращаем из метода параметризованное значение, то можно указать просто <?>. В таком случае мы не будем иметь никакой информации о типе, кроме того, что это Object.

\begin{lstlisting}
    public static void printAll(List<?> list) {
        for (Object o : list) {
            System.out.println(o);
        }
    }
\end{lstlisting}

Wildcards могут использоваться вместе с ограничениями super, extends. Рассмотрим пример структуры данных стек и две операции на ней - извлечь все элементы и поместить все элементы в стек.

\begin{lstlisting}
public class GenericStack<E> {
    // ...
    
    // push all elements from @src to stack
    public void pushAll(Collection<? extends E> src) {
        for (E e : src) {
            push(e);
        }
    }

    // pop all elements from stack to @dst
    public void popAll(Collection<? super E> dst) {
        while (!isEmpty()) {
            dst.add(pop());
        }
    }
}
\end{lstlisting}

В методе pushAll(Collection<? extends E> src) коллекцию src можно рассматривать как поставщика данных (producer). Поставщик может предоставлять элементы дочерних типов E. 

В методе popAll(Collection<? super E> dst) коллекция dst - потребитель данных (consumer). Потребитель может принимать значения супертипов E. 

\subsection*{PECS}
PECS - Producer Extends, Consumer Super - мнемоническое правило для написания правильных параметризованных методов для обработки коллекций.

Producer Extends - если нужна коллекция, которая поставляет данные (то есть, из нее читают), объявите ее <? extends T> 

ConsumerSuper - если нужна коллекция, которая потребляет данные (то есть, в нее пишут), объявите ее <? super T>

\subsection*{Наследование}
\begin{lstlisting}
List<String> strList = new ArrayList<String>();  
List<Object> objList = strList; // Compile time error
List rawList = strList; // OK - using raw type
\end{lstlisting}

Несмотря на то, что String является наследником Object, List<String> не является наследником List<Object>. 

Иерархия для ArrayList<String> -> List<String> -> Collection<String> -> Object
\begin{lstlisting}
List<String> strList = new ArrayList<String>();  
Collection<String> col = strList;
Object obj = col;
\end{lstlisting}


\newpage
\subsection*{Задачи}
\begin{enumerate}
    \item Дан интерфейс Stack. 
    \begin{lstlisting}
public interface Stack<E> {
    // add new element to the top of the stack
    public void push(E element) throws StackException;
    // return and remove an element from the top
    public E pop() throws StackException;
    // return the top element but doesn't remove
    public E peek();
    public int getSize();
    public boolean isEmpty();
    public boolean isFull();
    // add all elements from @src to the stack
    public void pushAll(Collection<? extends E> src) throws StackException;
    // pop all elements from stack to @dst
    public void popAll(Collection<? super E> dst) throws StackException;
}
    \end{lstlisting}
    Реализуйте класс class GenericStack<E> implements Stack<E>. Также реализуйте класс StackException extends Exception - исключение должно кидаться, если стек полон и кто-то вызывает метод push() или если стек пуст и кто-то вызывает pop().

    \item Есть коллекция элементов String. Вам нужно сортировать эту коллекцию по длине строки. Используйте Collections.sort(List<T> list, Comparator<? super T> cmp). 
    \begin{lstlisting}
// You need to implement LengthComparator implements Comparator
public static void test() {
    List<String> s = Collections.asList{"aaa", "b", "cd"};
    // Should return {b, cd , aaa}
    Collections.sort(s, new LengthComparator()); 
}

    \end{lstlisting}
    
    
    
    \item (*) Напишите тесты к своим решениям, использую библиотеку JUnit
    
    \item Финансовый менеджер. Приложение для контроля своих расходов. Каждый пользователь имеет логин и пароль. У пользователя могут быть несколько счетов. Каждый счет имеет текстовое описание, текущий остаток и список транзакций. У транзакции есть метка пополнение-снятие, дата проведения, сумма и описание. Нужно реализовать классы User, Account, Record и интерфейс DataStore. 
    
    Имя пользователя должно быть уникальным, каждый счет и транзакция также имеют уникальный номер. Вы должны уметь сравнивать два объекта (посмотрите, что такое equals(), hashCode()). 
    
    (*) Напишите comparator, который сортирует список транзакций по дате
    
    (*) Напишите тесты к своей реализации
    
    \begin{lstlisting}
public interface DataStore {
    // return null if no such user
    User getUser(String name);
    // If no users, return empty collection (not null)
    Set<String> getUserNames();
    // If no accounts, return empty collection (not null)
    Set<Account> getAccounts(User owner);
    // If no records, return empty collection (not null)
    Set<Record> getRecords(Account account);
    void addUser(User user);
    void addAccount(User user, Account account);
    void addRecord(Account account, Record record);
    // return removed User or null if no such user
    User removeUser(String name);
    // return null if no such account
    Account removeAccount(User owner, Account account);
    // return null if no such record
    Record removeRecord(Account from, Record record);
}
    \end{lstlisting}


    
    
\end{enumerate}
    


\end{document}